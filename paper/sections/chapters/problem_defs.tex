\section{Problem Definitions} \label{ch:defs}
In an effort to be consistent in naming conventions and parameter definitions throughout this work, the problem space will be defined. The conventions described in Section \ref{ch:defs} will be used throughout the rest of the work.

\subsection{The Field}
The initially unknown field, referred to as the \textit{target field}, will be a rectangular field of height $h$, and width $w$, i.e. $Z \in \mathbb{R}^{h \times w}$. The field is made up of square pixel cells, referred to as \textit{vesicles}. Each vesicle can be ``visited'', or sampled, in order to yield a number in the set of real numbers. Throughout this thesis, a square target field (i.e. $h = w$) will be used.

\subsection{The Sensor} \label{sec:sensor_measurements}
For the sake of a simpler introduction to the methods described in this chapter, the basis of the predictions will be observations of interest made using ideal sensors with no measurement noise. The sensors will measure a subset of the area of the entire target field. This area will be referred to as the \textit{sensor footprint}, and will be equal to the size of a single vesicle of the target field, $a$.

For the methods developed, the locations of the sensor measurements must be known. The locations will be represented as Cartesian coordinates on the field $Z$. For an arbitrary observation of the field $Z$, the location of the measurement will be at coordinates $\vect{s} \in \mathbb{R}^2$, and the corresponding sensor measurement would be $Z(\vect{s})$.

\subsubsection{Real World Sensing Examples}
If a Global Positioning System (GPS) sensor is used to estimate position, a Haversine Transformation would likely be used to convert Earth longitude and latitude to Cartesian coordinates within the target field. In the case of predicting the current location of a wildfire, for example, an infrared sensor would likely be used to measure the values of interest, thermal output of the field in this case. In the case of terrain mapping, for example, a lidar sensor might be used to sample terrain altitudes of the terrain below, at marked locations using GPS, on a UAV.
