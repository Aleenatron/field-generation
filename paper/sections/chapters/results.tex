\chapter{Results}

\section{Comparing The Method}
% talk about the zig zag (w/ & wo/ kriging predictions)
% compared against the zig zag

\subsection{Varying Target Field Sizes}
The effectiveness of the method introduced varies based on the area of the target field being explored. For small fields, a naive zig-zag exploration may be more efficient and less computationally expensive. By modulating the dimensions of the target field, a comparison can be drawn demonstrating the effectiveness of the method versus a naive zig-zag approach.

\begin{table}
\centering
	\begin{tabular}{ |p{3cm}||p{1.5cm}|p{1.5cm}|p{1.5cm}|p{1.5cm}|p{2cm}|  }
		\hline
		\multicolumn{6}{|c|}{Average Number of Waypoints Selected ($n=20$)} \\
		\hline
		Method & $50 \times 50$ & $100 \times 100$ & $250 \times 250$ & $500 \times 500$ & $1000 \times 1000$ \\
		\hline
		Kriging Explorer   	& 10 & 100 & 200 & 400 & 100 \\
		Zig-Zag Explorer	& AX & ALA & 248 & 300 & 100 \\
		\hline
	\end{tabular}
	\caption{Comparing the average number of waypoints selected between the two methods for varying field sizes.}
    \label{tab:path_wps}
\end{table}

\begin{table}
\centering
	\begin{tabular}{ |p{3cm}||p{1.5cm}|p{1.5cm}|p{1.5cm}|p{1.5cm}|p{2cm}|  }
		\hline
		\multicolumn{6}{|c|}{Average Path Length ($n=20$)} \\
		\hline
		Method & $50 \times 50$ & $100 \times 100$ & $250 \times 250$ & $500 \times 500$ & $1000 \times 1000$ \\
		\hline
		Kriging Explorer   	& 10 & 100 & 200 & 400 & 100 \\
		Zig-Zag Explorer	& AX & ALA & 248 & 300 & 100 \\
		\hline
	\end{tabular}
	\caption{Comparing the average path lengths between the two methods for varying field sizes.}
    \label{tab:path_lengths}
\end{table}

\begin{table}
\centering
	\begin{tabular}{ |p{3cm}||p{1.5cm}|p{1.5cm}|p{1.5cm}|p{1.5cm}|p{2cm}|  }
		\hline
		\multicolumn{6}{|c|}{Average Path Length ($n=20$)} \\
		\hline
		Method & $50 \times 50$ & $100 \times 100$ & $250 \times 250$ & $500 \times 500$ & $1000 \times 1000$ \\
		\hline
		Kriging Explorer   	& 10 & 100 & 200 & 400 & 100 \\
		Zig-Zag Explorer	& AX & ALA & 248 & 300 & 100 \\
		\hline
	\end{tabular}
	\caption{Comparing the average path lengths between the two methods for varying field sizes.}
    \label{tab:path_lengths}
\end{table}

\begin{table}
\centering
	\begin{tabular}{ |p{3cm}||p{1.5cm}|p{1.5cm}|p{1.5cm}|p{1.5cm}|p{2cm}|  }
		\hline
		\multicolumn{6}{|c|}{Average Loss of Prediction Variance ($n=20$)} \\
		\hline
		Method & $50 \times 50$ & $100 \times 100$ & $250 \times 250$ & $500 \times 500$ & $1000 \times 1000$ \\
		\hline
		Kriging Explorer   	& 10 & 100 & 200 & 400 & 100 \\
		Zig-Zag Explorer	& AX & ALA & 248 & 300 & 100 \\
		\hline
	\end{tabular}
	\caption{Comparing the average loss of prediction variance between the two methods for varying field sizes.}
    \label{tab:path_lengths}
\end{table}


% table :
% zig zag w/o krig predictions - zig zag w/ krig predictions - chasing the dragon (going to the max variance) - my method

% compare
% arc length (amount traversed supposed to represent energy expenditure and time)
% RMS error
% number of waypoints (not carrots) set
% rate of change of variance drops

% 20x runs of each and get average from each
% for fields 50x50, 100x100, 250x250, 500x500, 1000x1000
% show final pictures of at least one of each method

\subsection{Varying Autocorrelation}
% table :
% zig zag w/o krig predictions - zig zag w/ krig predictions - chasing the dragon (going to the max variance) - my method

% compare
% arc length (amount traversed supposed to represent energy expenditure and time)
% RMS error
% number of waypoints (not carrots) set
% rate of change of variance drops

% 20x runs of each and get average from each
% for autocorrelations 2, 4, 8, 16, 32

% show final pictures of at least one of each method