
\chapter{Introduction}
% Why is this problem important? 
% We are introducing the concept of aerial field exploration
Field exploration is a method in which an unknown field (a \textit{target field}) is learned in an attempt to discover traits or track trends about the field. Field exploration methods can be useful for tracking the health of crop soil, the size of ice glaciers, generating terrain maps, and a wide variety of scientific, agricultural, and industrial purposes. Furthermore, an exploration technique, versus a patrolling or tracking technique where a target is tracked or surveilled, does not require a model of the target field dynamics, as they can be learned on-the-fly.

%% for some real life examples, we are going to argue why UAVs are the medium in which we should choose for our explorations
Using an autonomous exploration vehicle, an unknown field of interest can be scanned within a more reasonable time frame, and cost, when compared to conventional scanning techniques involving satellite and manned-airplane missions. Using the techniques introduced, a high-quality map can be generated of a previously unknown field of interest on demand. Satellite imagery of Earth has been used for measuring various natural phenomena in the past several decades. Estimating polar ice cap melting rates and exploring the location of an oil spills are among the class of problems solved by exploration techniques. The US Forest Services' Moderate Resolution Imaging Spectroradiometer (MODIS) Active Fire Mapping Program updates images every one to two days with a fixed sensor payload in orbit. While this program is helpful for detecting large events with long periods of activity, the sampling rate of this service might not give an emergency response team or a scientist the required resolution and precision in gathered data at their desired rate. The resolution and frequency problem along with the cost associated with building, launching, and maintaining an orbiting Earth satellite might make some areas of research prohibitive. The use of unmanned aerial vehicles (UAVs), and other autonomous vehicles, have more recently been used in similar fields of study, industry, agriculture, and in environmental protection. The benefit in using these vehicles is more rapidly acquired data with adjustable accuracy. A UAV, for example, can give more nuanced and detailed data on features of a field that are not observable from the distance or field of view of an orbiting satellite with a fixed payload. The autonomous vehicle can be equipped with any flyable or drivable sensor that can be deployed from virtually anywhere to explore anywhere within reach.

% What have others done before me? How is what I'm doing different? Why is it better?
A common approach to exploring a field is to conduct a zig-zag pattern, or other predetermined maneuver on a target field. This task might take longer than needed to collect the required data, and could potentially ineffectively use the flight or drive time of the exploration vehicle which often has a limited runtime. A method that utilizes the learned stochastic properties of a field could be used to decrease exploration time by avoiding the need to scan more points than needed. A majority of the unobserved points in a field can be predicted to a known degree of confidence, given a degree of spatial autocorrelation in the field. Furthermore, scanning every point in a large unknown field is an unrealistic expectation for vehicles with limited maneuvering capabilities. This is especially a problem if the field as a whole is very large and needs to only be predicted to a small degree of confidence. A scheme for minimal and high-quality scanning via variance suppressing path planning would be in the benefit of time for the user(s) of such a system, and the scanning equipment as well.

Using the Kriging variances generated from a set of samples taken on a field, variance based path planning methods can be used to steer an exploration vehicle in the areas of maximal uncertainty, while traversing over  areas of low prediction confidence. The methods introduced attempt to help a user of this system explore an unknown field with a known degree of confidence that is configurable through a desired runtime, tuned by the user.

\section{Previous Works} \label{sec:prev_works}
The goal of this thesis is to introduce path planning techniques which reduce overall uncertainty of Kriging field predictions by steering a single vehicle through a field optimally. We use Kriging predictions as feedback into a path planner to estimate confidence return for a given candidate trajectory.

Exploration is a subset of the types of missions UAVs have been used for recently. From Section 2 of Nikhil Nigam's \textit{The Multiple Unmanned Air Vehicle Persistent Surveillance Problem: A Review} \cite{nigam:missions}, the various types of missions possible are described. There exist problems of tracking and patrolling which involve following a moving target, or of finding the spread rate and source of an item of interest. The exploration mission type is a procedure which runs parallel to the these types of missions. Without a model describing the states of the item of interest being explored, a simple scanning procedure involving random movements or following a predetermined path, like a zig-zag about the field as in \cite{semsch:uav_zig} are executed, or a zig-zag which incorporates the model dynamics of the vehicle, as in \cite{nigam:zigzag}. In \textit{Autonomous Aeromagnetic Surveys Using a Fluxgate Magnetometer} by Douglas G. Macharet et al., A UAV is used in a mineral field exploration technique, where a fluxgate sensor is used to measure the magnetic flux of a point beneath the UAV \cite{macharet:magnet}. A zig-zag pattern is ultimately used to explore the field for minerals of interest. A more dynamic strategy is used in the autonomous home vacuum cleaner \textit{Roomba} by iRobot, where a spiral pattern is used in an attempt to clean up and find the periphery of debris \cite{roomba:spiral}. The radius of the spiral pattern is a function of the amount of debris tracked by the debris sensor in the immediate area of the vacuum cleaner.

Exploration missions often do not specify the model of the item of interest being tracked. Knowing the model and kinematics of the item being tracked makes it possible to use an optimal estimation tool such as an Extended Kalman Filter as in Rabinovich et al. \textit{A Methodology For Estimation of Ground Phenomena Propagation} \cite{sharon:uav_est} and \textit{Multi-UAV Path Coordination Based on Uncertainty Estimation} \cite{sharon:uav_uncert} where the velocity and position states of a ground fire are estimated while tracking the points surrounding the periphery of a wildfire. The planner for this mission calculates a path based on the Kalman variances of the control points representing the periphery of the ground phenomenon being tracked. Variance suppressing path planning is used in the path planners introduced in this thesis in Chapter \ref{ch:pp}.

The Kriging Method has been used in a UAV Contour Tracking problem in Zhang et al. \textit{Oil Spills Boundary Tracking Using Universal Kriging And Model Predictive Control By UAV} \cite{zhang:oil_krig}. The work relies on the knowledge of a model of the oil spill, and therefore is not a generic case of an exploration problem of a model-less field.

C. C. Castello et al. present the use of the Kriging method for environmental sensor placement in \textit{Optimal Sensor Placement Strategy for Environmental Monitoring using Wireless Sensor Networks} \cite{kriging:sensorplacement}. The overall variances of a Kriging predicted field, predicted from a set measurements from fixed sensor locations, can be directly compared to the variances of predicting the same field with a different set of sensor locations. The method can therefore be used to help assist in optimal sensor placement by conducting a Monte Carlo simulation of random sensor placements, and ultimately choosing the random configuration that minimized the Kriging prediction variances for the field. A path planner, which can be stated as a sensor placement problem, by selecting a random path, or set of sampling locations, that minimizes the expected Kriging variance of a target field is introduced in this thesis in Section \ref{sec:mcpp}. The use of a Monte Carlo approach, where noise is used to assist in suppressing prediction uncertainty has been used for uncertainty suppression in obstacle avoidance motion planning in \textit{Monte Carlo Motion Planning for Robot Trajectory Optimization Under Uncertainty} \cite{janson:mcmp}, but the technique was not used for exploration purposes, as introduced in this thesis.

%% Their stuff...
The use of Kriging variances in a path planner in a publication, when this thesis was originally started, did not exist. Near the completion of this thesis, a paper discussing the benefits of using Kriging variance motivated path planning for field exploration, was published \cite{fentanes:soilkrig}. Pulido Fentanes et al. published \textit{Kriging-Based Robotic Exploration for Soil Moisture Mapping Using a Cosmic-Ray Sensor}, where a Kriging variance based exploration technique is used for the purpose of quality mapping of agricultural soil moisture \cite{fentanes:soilkrig}. In the publication, three Kriging, variance motivated, path planners are used to reduce Kriging prediction error by steering an agricultural robot into areas of high Kriging variance calculated by using all data gathered on a path. 

The first of their path planners, named \textit{Greedy Next-Best-View} (NBV), similar to the \textit{Highest Variance} (HV) strategy demonstrated in this thesis, simply targets the highest variance on the Kriging variance field. In the Greedy NBV algorithm, the path is recalculated every time the robot takes a sample. In HV, a path is only recalculated when the last endpoint (the highest variance of the last Kriging calculation of the field) is met. The authors of the publication also introduce a \textit{Monte Carlo Next-Best-View} where a set of random endpoints are generated and weighted against one another according to their Kriging variances. The endpoint with the highest uncertainty is selected as the next sample location. 

In this thesis, a Monte Carlo technique is also introduced in Section \ref{sec:mcpp}, but instead of weighing each proposed random trajectory by its Kriging variance, the predicted values of the points along a random trajectory are resused, and a Kriging variance calculation is run on the field again. The path that is ultimately chosen, in this planner proposed, is the path which reduces the expected overall Kriging variance of the field as a whole. Lastly, the authors introduce an adaptive sampling planner which works by generating an initial path that is then modified after each sample taken. As more possible path are generated randomly, points get removed from the possible set of endpoints when their Kriging variance falls below the mean of the variance field. This method considers the mission time and minimum expectation of the measurement quality by re-planning and using a Traveling Salesperson (TSP) algorithm. In this thesis, minimum expected measurement quality is set before an exploration by tuning the maximum allowed area to scan.
\chapter{Problem Definitions} \label{ch:defs}
In an effort to be consistent in naming conventions and parameter definitions throughout this work, the problem space will be defined. The conventions described in Section \ref{ch:defs} will be used throughout the rest of the work.

\section{The Field}
The initially unknown field, referred to as the \textit{target field}, will be a rectangular field of height $h$, and width $w$, i.e. $Z \in \mathbb{R}^{h \times w}$. The field is made up of square pixel cells, referred to as \textit{vesicles}, of unit area $a \in \mathbb{R}$. The target field is \textit{gridded} in order to discretize the target field as a whole. Each vesicle can be "visited" or sampled in order to yield a number in the set of real numbers. Throughout the work presented, $a$ will be constant throughout the field, and the total area of the target field will be $A = ahw$. Throughout this work, a square target field (i.e. $h = w$) will be used.

\section{The Sensor} \label{sec:sensor_measurements}
For the sake of a simpler introduction to the methods described in this chapter, the basis of the predictions will be observations of interest made using ideal sensors with no measurement noise. The sensors will measure a subset of the area of the entire target field. This area will be referred to as the \textit{sensor footprint}, and will be equal to the size of a single vesicle of the target field, $a$.

For the methods developed, the locations of the sensor measurements must be known. The locations will be represented as Cartesian coordinates on the field $Z$. For an arbitrary observation of the field $Z$, the location of the measurement will be at coordinates $\vect{s} \in \mathbb{R}^2$, and the corresponding sensor measurement would be $Z(\vect{s})$.

\subsection{Real World Sensing Examples}
If a Global Positioning System (GPS) sensor is used to estimate position, a Haversine Transformation would likely be used to convert Earth longitude and latitude to Cartesian coordinates within the target field. If predicting the current location of a wildfire, for example, an infrared sensor would likely be used to measure the values of interest, thermal output of the field in this case.

