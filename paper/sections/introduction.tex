
\chapter{Introduction}
% Why is this problem important? 
% We are introducing the concept of aerial field exploration
Aeriel field exploration is a method in which information from an unknown ground field, a \textit{target field}, can be collected in order to discover traits or track the trends about the field. An exploration method could be a portable generic method for field exploration where a model of the target field phenomena is not known. From a distance above the ground, traits of a target field can be measured using what ever sensor is portable enough to fly or orbit around a field of interest.

%% Here we are going to argue why UAVs are the medium in which we should choose for our explorations
Satellite imagery of Earth has been used for measuring various natural phenomena in the past several decades. Estimating polar ice cap melting rates and exploring the location of an oil spill are among the class of problems solved by this technology. Currently, using a service like the US Forest Services' Moderate Resolution Imaging Spectroradiometer (MODIS) Active Fire Mapping Program, images are updated every 1 to 2 days with a fixed sensor. While this program is helpful for detecting large events with long periods of activity, the sampling rate of this service might not give an emergency response team or a scientist the required resolution and precision in gathered data at their desired rate. The resolution and frequency problem along with the cost associated with building, launching, and maintaining an orbiting Earth satellite might even make some areas of research prohibitive. The use of unmanned aerial vehicles (UAVs) have more recently been used in similar fields of study and in environmental protection. The benefits gained from using UAVs is that of more rapidly acquired data with more easily adjusted accuracy. Furthermore, a UAV can give more nuanced and detailed data on features of a field that are not observable from the distance or field of view of an orbiting satellite. This is because the UAV can be equipped with any compatible sensor and can be deployed from virtually anywhere to fly virtually anywhere. Furthermore, an exploration technique, versus a patrolling or tracking technique, does not require a model of the target field dynamics.

% What have others done before me? How is what I'm doing different? Why is it better?
Presently, scanning fields with unmanned aerial vehicle systems is a task involving flying one or more UAVs in a zig-zag, or other predetermined, maneuver above a target field, while collecting data. This task might take longer than needed to collect the required data, and could potentially ineffectively use the flight time of the UAV which often has a short and limited flight time. Furthermore, scanning every point in a large unknown field is an unrealistic expectation for many UAVs. This is especially a problem if the field as a whole needs is very large and needs to only be explored to a small degree of confidence. A scheme for minimal scanning and effective path planning would be in the benefit of time of the user(s) of the system, and the flown equipment as well. 

% How is what I'm doing different? Why is it better? (part II)
Predictions of the field states must therefore be computed from only a finite set of measurements at known locations. The confidence of predictions should also be attainable through the scheme, as they will be used as metrics to determine the confidence of the predictions. Using a prediction based method to fill in the unseen gaps, though quicker than scanning every point, does not take advantage of the statistical properties of the target field to attempt to minimize flight time and path taken. 

Due to the nature of much of the phenomena one might be interested in scanning in an unknown field, a method that exploits the known stochastic and statistical properties of the field could be used to decrease flight time. A field that exhibits properties of \textit{geospatial autocorrelation}, will be more statistically exploitable to find holes in confidence which can be used to help maneuver the UAV in a direction of low prediction confidence. The Kriging Method, a popular interpolation tool, offers a prediction and a variance of prediction. We can exploit the Kriging variances generated by our prediction to motivate the UAVs to autonomously steer in the areas of least confidence, while traversing over other areas of low prediction confidence, until a minimum confidence in prediction is achieved for an entire unknown target field. The methods introduced attempt to help a user of this system explore an unknown field to a desired level of confidence in a short period of time, using a more methodical path, relative to current exploration techniques. 
