\section{Previous Work}
The goal of this thesis is to introduced path planning techniques which reduce overall uncertainty of Kriging field predictions by steering a single vehicle through a field. The vehicle should optimally sample vesicles in a field in order to create a map of predetermined quality. Using the next expected overall Kriging variance of a field, after taking a given path, using the Kriging method has not been directly used. Using Kriging predictions as feedback into a path planner to estimate confidence return for a given trajectory is introduced in this work.

Field scanning techniques, field prediction, and unknown field exploring are active research areas. Several publications in the past decade alone have covered these types of missions, and demonstrated results using UAVs.

Exploration is a subset of the types of missions UAVs have been used for recently. From Section 2 of Nikhil Nigam's \textit{The Multiple Unmanned Air Vehicle Persistent Surveillance Problem: A Review} \cite{nigam:missions}, the various types of missions possible are described. There exist problems of tracking and patrolling which involve following a moving target, or of finding the spread rate and source of an item of interest. The \textit{exploration} mission type is a procedure which runs parallel to the these types of missions. In \textit{Autonomous Aeromagnetic Surveys Using a Fluxgate Magnetometer} by Douglas G. Macharet et al., A UAV is used in a mineral field exploration technique, where a fluxgate sensor is used to measure the magnetic flux of a vesicle beneath the UAV \cite{macharet:magnet}. A zig-zag pattern is ultimately used to explore the field for minerals of interest. A more dynamic strategy is used in the autonomous home vacuum cleaner \textit{Roomba} by iRobot, where a spiral pattern is used in an attempt to clean up debris once found \cite{roomba:spiral}. The radius of the spiral pattern is a function of the amount of debris tracked by the debris sensor in the immediate area of the vacuum cleaner.

Exploration missions often do not specify the model of the item of interest being tracked. Knowing the model and kinematics of the item being tracked makes it possible to use an optimal estimation tool such as an Extended Kalman Filter as in Rabinovich et al. \textit{A Methodology For Estimation of Ground Phenomena Propagation} \cite{sharon:uav_est} and \textit{Multi-UAV Path Coordination Based on Uncertainty Estimation} \cite{sharon:uav_uncert} where the velocity and position states of a ground fire are estimated while tracking the points surrounding the periphery of a wildfire.

In the case of tracking a non-dynamic state of interest, or an unknown field model, a statistical model can be first generated, and the hidden states can be predicted from the learned distributions. Often times, exploration missions are not searching for anything in particular, but rather exploring for the sake of discovery. Without a model describing the states of the item of interest being explored, a simple scanning procedure involving random movements or following a predetermined path, like a zig-zag about the field as in \cite{semsch:uav_zig} are executed, or a zig-zag which incorporates the model dynamics of the vehicle, as in \cite{nigam:zigzag}.

The Kriging Method has been used in a UAV Contour Tracking problem in Zhang et al. \textit{Oil Spills Boundary Tracking Using Universal Kriging And Model Predictive Control By UAV} \cite{zhang:oil_krig}. The paper relies on the knowledge of a model of the oil spill, and therefore is not a generic case of an exploration problem.

The use of a Monte Carlo approach, where noise is used to assist in suppressing prediction uncertainty has been used for uncertainty suppression in obstacle avoidance motion planning in \textit{Monte Carlo Motion Planning for Robot Trajectory Optimization Under Uncertainty} \cite{janson:mcmp}, but the technique is not used for exploration purposes, as introduced in this thesis in section \ref{sec:mcpp}.

Whilst nearing completion on the development of the methods introduced in this thesis, Pulido Fentanes et al. published \textit{Kriging-Based Robotic Exploration for Soil Moisture Mapping Using a Cosmic-Ray Sensor}. A Kriging variance based exploration technique is used for the purpose of quality mapping of agricultural soil moisture \cite{fentanes:soilkrig}. In their work, they do things i do X,Y,Z, but they do not do A,B,C.

% Ren: "Are there any other references using UAV steering in exploration?"