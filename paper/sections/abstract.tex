
\begin{abstract}

A set of methods and path planners are introduced for the exploration of unknown semi-to-fully-ergodic fields of interest. The spatial statistical properties of a target field can be exploited to assist in variance suppressing planning techniques from observations of a single state of interest. The Kriging Method, a \textit{Best Linear Unbiased Predictor}, is used to exploit the statistical properties, namely the spatial autocorrelation, of a target field. The Kriging Method predicts the state of unobserved points from a set of observed points for the purposes of quality mapping. A prediction and confidence of prediction of the entirety of a given target field can be generated from the method.

The path planners introduced can be used to reduce the overall prediction uncertainty of a field by steering a single vehicle to collect a good set of samples. A metric for return on investment of executing a trajectory using feedback from Kriging predictions is presented. The five path planners introduced suppress the overall uncertainty of a Kriging prediction of an unknown target field in order to create a higher quality map when compared to a preplanned scanning regime, and another Kriging variance suppressing method (Greedy Next-Best-View), for the same distance traveled.

\end{abstract}
