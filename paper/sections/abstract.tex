
\begin{abstract}
Using an Unmanned Aerial Vehicle (UAV) system, a unknown field of interest can be scanned within a more reasonable time frame compared to conventional scanning techniques involving satellite and manned-airplane missions. Potentially more nuanced data can be gathered from the UAV made observations because of more desirable fields of view and more customizable sensors on-board.

The Kriging Method, a \textit{Best Linear Unbiased Predictor} (BLUP) commonly used in the field of Geostatistics, can be used to exploit the statistical properties, namely the geospatial autocorrelation of a target field, to better predict unobserved points from a set of observed points. Using a modified Universal Kriging Method (for ``on the fly" use ), a prediction and confidence of prediction of the entirety of a given target field can be generated from a set of measurements. From the variances associated with the predictions by the Kriging Method, a corresponding path-planner for UAV field exploration can be used to reduce the overall uncertainty of field predictions by steering a single vehicle in the direction of the highest uncertainty in prediction.

A method for the exploration of unknown semi-to-fully-ergodic fields of interest using observations from an autonomous system using the Kriging Method is introduced. By tessellating a target scan-space into sub-fields with an associated level of prediction confidence, a graph representing the confidence of predictions in adjacent tessellations can be constructed and traversed to reduce the maximum uncertainty of prediction of the entire target space. An experimental simulation and comparisons to other techniques are presented. 

The method introduced is portable in nature, and can be used to explore a variety of fields, and its ability is not limited to observing and exploring with UAVs. The method along with the cost-effective nature of a UAV make the combination attractive to those studying phenomena that were otherwise difficult to explore, and can improve current methods of research, exploration, and natural emergency response.

\end{abstract}
