
\begin{abstract}
Using an Unmanned Aerial Vehicle (UAV) system, a field of interest can be scanned within a more reasonable time frame compared to conventional scanning techniques involving sattelite and manned-airplane missions. Potentially more nuanced data can be gathered from the UAV made observations because of more desirable fields of view and more customizable sensors on-board.

The Kriging Method, a \textit{Best Linear Unbiased Predictor} (BLUP) commonly used in the field of Geostatistics, exploits the statistical properties of natural phenomena to better predict unobserved points from a set of observed points. Using a modified Universal Kriging Method (for ``on the fly" use ), a prediction and confidence of prediction of the entirety of a given target field can be generated from a set of measurements. From the variances associated with the predictions by the Kriging Method, a corresponding path-planner for UAV field exploration is can be used to reduce the overall uncertainty of field prediction.

A method for the exploration of semi-ergodic fields of interest using aerial observations from UAV using the Kriging Method is introduced. By tessellating a target scan-space into sub-fields with an associated level of prediction confidence, a graph representing the confidence of predictions in adjacent tessellations can be constructed and traversed to reduce the maximum uncertainty of prediction of the entire target space. An experimental simulation and comparisons to other techniques are presented. The method along with the cost-effective nature of a UAV can improve current methods of research, exploration, and natural emergency response. 
\end{abstract}
