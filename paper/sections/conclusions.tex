\chapter{Conclusion}
The potential in a procedure using the Kriging Method as the core of a field exploration technique with an autonomous vehicle was demonstrated. By characterizing the confidence of the Kriging predictions made from observations in a field, along with uncertainty suppressing motivated path planners, the overall confidence in prediction of a target field as a whole can be maximized without having to scan every point. Furthermore, variance motivated path planning for prediction uncertainty suppression was shown to yield lower prediction errors and prediction variances for equal path lengths compared to a predetermined path for fields with a reasonable degree of spatial autocorrelation.

An exploration vehicle could be maneuvered through a field to collect samples in areas of low Kriging prediction confidence. This in turn can increase the quality of prediction of the target field's state of interest to a higher degree of certainty. Path planning techniques using Kriging variance suppression can produce a more accurate prediction of a target field while exploring with the same path length as a preplanned zig-zag scan pattern.

For highly spatially autocorrelated fields, with factor $\sigma_{field} = 100$ (Section \ref{sec:sigma100}), the Monte Carlo Path Planner (Section \ref{sec:mcpp}) performed better when compared to the other path planners demonstrated in terms of reducing field prediction error. The results were closely followed by the $N$-HV and RGA methods, and then the zig-zag method. The preplanned zig-zag method performed better than the HV method. All methods outperform the $30\%$ zig-zag method up to the $20\%$ scan mark for high to mid autocorrelation factors. For a field with a low spatial autocorrelation factor of $\sigma_{field} = 1$ (Section \ref{sec:sigma1}), the preplanned zig-zag methods ($20\%$ and $30\%$ scan limited zig-zags) performed the best in terms of reducing prediction error past the $10\%$ scan mark. This is due to the planners ability to scan a more evenly distributed path along the field. A more evenly distributed path across the field implies more spatial characteristics are known about the field, and therefore make the field more predictable.

\chapter{Future Work}
Future work can be done in an effort to further develop Kriging prediction variance motivated path planning techniques. A comparison of the introduced methods for different vehicle dynamics, e.g. a Dubins Vehicle, can be conducted to show the effectiveness of the introduced methods. Additionally, an implementation of these methods on flying and/or driving hardware can be developed to demonstrate the methods and their differences in a non-simulated setting.

A modification can be made to the Monte Carlo Path Planner and the $N$-HV method to create trajectories by amending the best waypoints along the way to a selected decision point. For each waypoint selected, along a leg, the trajectory computed up to a waypoint can be fed back into the prediction and variance calculation process from the predicted points on the trajectory (similar to the current methods for a whole leg). The trajectory, up to that waypoint, can then be compared to a calculated trajectory up to a neighboring waypoint. The trajectory that is considered more optimal up to the candidate waypoint, will qualify to the next phase of waypoint selection. The process will continue until the final intended decision point has been met. This will in turn produce a theoretically more optimal path over the current methods, but at a much higher computational cost.

A method using a combination of the planners introduced and a preplanned trajectory can be done by switching the exploration planning method dynamically based on the autocorrelation range of the field. When the autocorrelation of a region of the field is considered to have low spatial autocorrelation, a preplanned trajectory can be used to explore that section of the field, and another variance suppressing method can be used to explore the regions on the field with higher spatial autocorrelation factors.

Further work can attempt to minimize overall Kriging variance for multiple states of interest across the field while simultaneously predicting more than one state of interest. This can be done by weighing the cost of predicting each of the states of interest dynamically based on the current overall variances of each of the state predictions on the field.
