\chapter*{Conclusion}
The potential in a procedure using the Kriging Method as the core of a field exploration technique with an autonomous vehicle was demonstrated. By characterizing the confidence of the Kriging predictions made from observations in a field, along with uncertainty suppressing motivated path planners, the overall confidence in prediction of a target field as a whole can be maximized without having to scan every point. Furthermore, variance motivated path planning for prediction uncertainty suppression was shown to yield lower prediction errors and prediction variances for equal path lengths compared to predetermined paths.

An exploration vehicle could be maneuvered through a field to collect samples in areas of low Kriging prediction confidence. This in turn can increase the quality of prediction of the target field's state of interest to a higher degree of certainty. Path planning techniques using Kriging variance suppression can produce a more accurate prediction of a target field while exploring with the same path length as a preplanned zig-zag scan pattern.

TODO: draw conclusions from results

\chapter*{Future Work}
Future work can be done in an effort to further develop Kriging prediction variance motivated path planning techniques. A comparison of the introduced methods for varying vehicle dynamics, e.g. a Dubins Vehicle, can be conducted to show the effectiveness of the introduced methods. Additionally, an implementation of these methods on flying and/or driving hardware can be developed to demonstrate the methods and their differences in a non-simulated setting.

% Firstly, a method which filters noisy measurements on a target field, with potentially dynamic properties, using a Kalman Filtering technique may be in the interest of a variance motivated path planner. The field states of a potentially dynamic field can be estimated using a Kalman Filter, as done in \textit{Toward Dynamic Monitoring and Suppressing Uncertainty in Wildfire by Multiple Unmanned Air Vehicle System} by Sharon Rabinovich et al. \cite{sharon:jor}. The variances from both the Kalman filter state estimates, along with the Kriging prediction variances, could then potentially be fused in an effort to better track the states of a dynamic field by constructing paths along the field that suppress overall Kalman filter variances and Kriging prediction variances. This would ideally, in turn, produce predictions with higher certainty of state estimate for the overall field. The method could then also be used to help determine if a vehicle is suitable for tracking a certain state about the field. For example, if the dynamics of the field are found to be too dynamic for a simulated vehicle with a fixed velocity to track, while minimizing Kriging prediction variances and Kalman state variances, a more suitable vehicle might be chosen instead for that field. Furthermore, the Kriging state predictions and prediction variances could serve as the initial prediction and variance values for the Kalman Filter.

% Further work could also be the development of a comparison of the introduced methods, for varying vehicle dynamics, e.g. a Dubins Vehicle.